\documentclass{res}

%\usepackage{helvetica} % uses helvetica postscript font (download helvetica.sty)
%\usepackage{newcent}   % uses new century schoolbook postscript font 
\usepackage{verbatim}
% \usepackage{biblatex-chicago}

\usepackage{bibentry}
 
\setlength{\textheight}{9.5in} % increase text height to fit on 1-page 

\usepackage{natbib}
\usepackage{bibentry}
\makeatletter\let\saved@bibitem\@bibitem\makeatother
\usepackage[colorlinks=true]{hyperref}
\makeatletter\let\@bibitem\saved@bibitem\makeatother

\hypersetup{
    colorlinks=true,
    linkcolor=blue,
    filecolor=magenta,      
    urlcolor=blue,
}

\begin{document} 
\newsectionwidth{15 pt}
% Center the name over the entire width of resume:
 \moveleft.5\hoffset\centerline{\large\bf Eli Ben-Michael}
 
% Draw a horizontal line the whole width of resume:
 \moveleft\hoffset\vbox{\hrule width\resumewidth height 1pt}


% address begins here
% Again, the address lines must be centered over entire width of resume:
\centerline{323 Evans Hall, Berkeley, CA 94720 }
\centerline{ (908)-472-6870 $\mid$ ebenmichael@berkeley.edu $\mid$ \href{http://github.com/ebenmichael}{github.com/ebenmichael} $\mid$ \href{ebenmichael.github.io}{ebenmichael.github.io}}

                                  
\begin{resume}

\section{EDUCATION}
    \textbf{University of California, Berkeley}, Berkeley, CA \hfill Expected May 2021\\
    PhD in Statistics, Advisor: Avi Feller, Research Topic: Causal Inference
    % \textbf{Research Interests}: Causal Inference, Machine Learning

    \textbf{Columbia University}, Columbia College, New York, NY \hfill May 2016\\
    Bachelor of Arts, \textit{Summa Cum Laude}, Computer Science and Statistics\\
    \textbf{Honors}: Phi Beta Kappa, Computer Science Department Award (top Computer Science Major), Dean's List (Fall 2012-Spring 2016)



 \makeatletter 
 \renewcommand\BR@b@bibitem[2][]{\BR@bibitem[#1]{#2}\BR@c@bibitem{#2}}           
 \makeatother
\bibliographystyle{apalike}
\nobibliography{my_papers}
\section{RESEARCH}
\textbf{Research Interests}:
\begin{itemize}
\item[] Causal Inference, Machine Learning, Econometrics% , Optimization
\end{itemize}%
\vspace{-4mm}
\textbf{Working Papers:}
\begin{itemize}
\item[] \bibentry{benmichael2018_ascm}
\end{itemize}
\vspace{-4mm}
\textbf{Contributed Talks:}
\begin{itemize}
\item[] UAI 2018 Causal Workshop, 2018 European Winter Meeting of the Econometric Society
% \item[] \bibentry{benmichael2018_uai}
% \item[] \bibentry{benmichael2018_eswm}
\end{itemize}
\vspace{-4mm}
\textbf{Open Source Software:}
\begin{itemize}
\item[] \href{https://github.com/ebenmichael/augsynth}{\texttt{augsynth}}: \texttt{R} implementation of the augmented synthetic control method
\end{itemize}
\section{EXPERIENCE}

\textbf{U.C. Berkeley Department of Statistics}, Berkeley, CA  \hfill{Fall 2017, Fall 2018}\\
\textit{Graduate Student Instructor}
\begin{itemize}
  % \item Lectured recitations and created assignments and assessments for Stat 159/259: Reproducible and Collaborative Data Science; the course focused on reproducible analysis using the Python data science stack and tools such as version control and Makefiles
% \item Lectured recitations and created assignments and assessments for
% \vspace{-1mm}  
%   \begin{itemize}
%   \item[] Stat 159/259: Reproducible and Collaborative Data Science     
%   \item[] Stat 232: Experimental Design
%   \end{itemize}
% \end{itemize}
% \item Lectured recitations and created assignments and assessments for Stat 159/259: Reproducible and Collaborative Data Science and Stat 232: Experimental Design
\item (Fall 2017) Stat 159/259: Reproducible and Collaborative Data Science
\item (Fall 2018) Stat 232: Experimental Design  
\end{itemize}  
\vspace{-2mm}  
\textbf{Walmart Labs}, Sunnyvale, CA \hfill{Summer 2017}\\
\textit{Machine Learning Scientist Intern}
\begin{itemize}
\item Designed models of consumer purchase behavior to learn latent representations of products
\item Implemented efficient learning algorithms on tens of millions of consumer purchases with Spark
% \item Assessed resulting representations' ability to reconstruct a human-generated catalog
\item Validated the representations' predictive power by reconstructing a human-generated catalog
  % \item Communicated results to various teams across the organization
\item Presented results to various teams across the organization for use in their modelling pipelines
\end{itemize}


\textbf{Knewton}, New York, NY \hfill{Summer 2016}\\
\textit{Data Science Intern}
\begin{itemize}
\item Generalized Bayesian models of student learning to incorporate hierarchical structure 
\item Scaled learning algorithms with a 10x speedup  using Spark
\item Analyzed performance, strengths, and weaknesses of models on student data
% \item Presented and documented models, code, and results
\end{itemize}

\begin{comment}
%BIG DATA
\textbf{Big Data Summer Institute}, University of Michigan, Ann Arbor, MI \hfill Summer 2015
\\ \textit{Intern}
\begin{itemize}
\item Implemented a search and counting algorithm for sub-sequences of the human genome
\item Attended lectures on machine learning, algorithms, data science, and statistics 
\item Presented work to faculty and undergraduates to promote the program to other institutions
\end{itemize}
\end{comment}

%ECON
\textbf{Columbia University Department of Economics}, New York, NY \hfill Fall 2014 - Spring 2016\\
\textit{Research Assistant}
\begin{itemize}
\item Built a natural language processing text analysis application in Python for use by economists 
\item Extracted text features from corpora using Python, NLTK, and gensim
\item Performed econometric and statistical analysis on text data with associated metadata
\end{itemize}


\begin{comment}
%DIMACS
	\textbf{Center for Discrete Mathematics and Theoretical Computer Science}, Piscataway, NJ\\
\textit{Research Intern} \hfill Summer 2014 
\begin{itemize}
\item Implemented a spectral machine learning algorithm in Python to learn a Hidden Markov Model
\item Improved algorithm runtime 100x through utilization of Python's NumPy and SciPy libraries
\item Presented results to colleagues and faculty to gain feedback
\end{itemize}

%TA
\textbf{Columbia University Departments of Mathematics and Statistics}, New York, NY
  \\ \textit{Undergraduate Teaching Assistant}\hfill Spring 2014 - Spring 2016
\begin{itemize}
\item Held regular office hours for calculus, linear algebra, and statistics students
\item Evaluated assignments and provided feedback to students
\end{itemize}
\end{comment}


% \section{EXTRACURRICULAR ACTIVITIES}
% 	\textbf{Columbia Data Science Society}, Vice President of Education \hfill Fall 2015 - Spring 2016
% 	\begin{itemize}
% 	\item Designed and taught workshops on data science tools such as Python and R
% 	\item Ran a committee of six people which organized educational events in data science
% 	\end{itemize}
% \begin{comment}	
% 	\textbf{Personal Projects} (On Github)
% 	\begin{itemize}
% 	\item Qlearning: Implementation of deep reinforcement learning algorithms
% 	\item brownian: Simulation of Brownian motion and general stochastic differential equations
% 	\item memm\_tagging: Implementation of a Maximum Entropy Markov Model 
% 	\end{itemize}
% \end{comment}

\section{SKILLS}
Observational Studies, Experimental Design, Quasi-Experimental Methods, Machine Learning, R, Python % SQL, Scala, Spark, MATLAB, Java


\end{resume}

\end{document}